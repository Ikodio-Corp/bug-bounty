\documentclass[12pt,a4paper]{article}
\usepackage[utf8]{inputenc}
\usepackage[indonesian]{babel}
\usepackage{geometry}
\usepackage{enumitem}
\usepackage{titlesec}
\usepackage{fancyhdr}
\usepackage{setspace}

\geometry{a4paper, margin=2.5cm}
\setstretch{1.15}

\pagestyle{fancy}
\fancyhf{}
\fancyhead[L]{\small Perjanjian Bootstrap - IKODIO Bug Bounty Platform}
\fancyhead[R]{\small \thepage}
\renewcommand{\headrulewidth}{0.4pt}

\titleformat{\section}{\normalfont\Large\bfseries}{Pasal \thesection.}{0.5em}{}
\titleformat{\subsection}{\normalfont\large\bfseries}{\thesubsection.}{0.5em}{}

\begin{document}

\begin{center}
\textbf{\LARGE PERJANJIAN BOOTSTRAP}\\[0.3cm]
\textbf{\Large POSISI: DATA ENGINEER, DATABASE ADMINISTRATOR DAN DEVOPS ENGINEER}\\[0.5cm]
\large Nomor: PB/IDK/2025/001\\[1cm]
\end{center}

Pada hari ini, tanggal 30 November 2025, telah dibuat dan ditandatangani Perjanjian Bootstrap oleh dan antara:

\vspace{0.5cm}

\noindent\textbf{PIHAK PERTAMA:}

\begin{tabular}{ll}
Nama Perusahaan & : PT. Intelligent Digital Knowledge \\
Alamat & : [Alamat Perusahaan] \\
NPWP & : [NPWP Perusahaan] \\
Diwakili oleh & : Hylmi Rafif Rabbani \\
Jabatan & : Chief Executive Officer (CEO) \\
\end{tabular}

\vspace{0.3cm}

Selanjutnya disebut sebagai \textbf{PEMBERI ROLE}.

\vspace{0.5cm}

\noindent\textbf{PIHAK KEDUA:}

\begin{tabular}{ll}
Nama Lengkap & : ABDUL RAOEF YUDHA QURNIA \\
NIK & : [NIK] \\
Tempat, Tanggal Lahir & : [Tempat], [Tanggal] \\
Alamat & : [Alamat Lengkap] \\
NPWP & : [NPWP] \\
\end{tabular}

\vspace{0.3cm}

Selanjutnya disebut sebagai \textbf{BOOTSTRAP MEMBER}.

\vspace{0.5cm}

PEMBERI ROLE dan BOOTSTRAP MEMBER secara bersama-sama disebut sebagai \textbf{PARA PIHAK}, dengan ini sepakat untuk mengadakan Perjanjian Bootstrap dengan syarat dan ketentuan sebagai berikut:

\section{PENGANGKATAN DAN JABATAN}

\subsection{Jabatan dan Penempatan}

PEMBERI ROLE dengan ini mengikutsertakan BOOTSTRAP MEMBER dan BOOTSTRAP MEMBER menerima role bootstrap pada PEMBERI ROLE dengan ketentuan sebagai berikut:

\begin{enumerate}[label=\alph*)]
\item Jabatan: Data Engineer, Database Administrator, dan DevOps Engineer
\item Penempatan: IKODIO Bug Bounty Platform Development Team
\item Pelaporan: Langsung kepada Chief Executive Officer (CEO)
\item Status: Bootstrap Member (masa bootstrap hingga produk launch)
\end{enumerate}

\subsection{Masa Bootstrap}

\begin{enumerate}[label=\alph*)]
\item BOOTSTRAP MEMBER akan menjalani masa bootstrap selama periode pengembangan hingga produk launching.
\item Selama masa bootstrap, PEMBERI ROLE berhak mengevaluasi kontribusi dan kinerja BOOTSTRAP MEMBER.
\item Setelah produk launching, akan dilakukan evaluasi untuk penentuan posisi dan kompensasi jangka panjang.
\end{enumerate}

\section{LINGKUP ROLE DAN TANGGUNG JAWAB}

BOOTSTRAP MEMBER bertanggung jawab atas empat area utama: Data Engineering, Database Administration, DevOps Operations, dan Data Management pada sistem IKODIO Bug Bounty Platform.

\subsection{Tanggung Jawab Data Engineering (90 Tugas)}

\subsubsection{ETL Pipeline Development dan Management}

\begin{enumerate}[label=\arabic*.]
\item Merancang dan mengembangkan ETL pipeline untuk data vulnerability, scan results, user activity, dan bug reports.
\item Mengimplementasikan data ingestion dari berbagai sumber: scanner outputs (Burp, ZAP, Nuclei), user submissions, API integrations.
\item Membangun data transformation logic untuk normalisasi vulnerability data dari format berbeda.
\item Mengoptimasi ETL performance untuk pemrosesan scan results real-time.
\item Mengimplementasikan error handling dan retry mechanisms pada ETL pipelines.
\item Membangun monitoring untuk ETL job success rates dan execution times.
\item Mengelola data quality checks pada setiap stage ETL pipeline.
\item Mengimplementasikan incremental loading untuk large datasets.
\item Membangun CDC (Change Data Capture) untuk tracking database changes.
\item Mengoptimasi ETL costs melalui efficient resource allocation.
\end{enumerate}

\subsubsection{Apache Airflow Orchestration}

\begin{enumerate}[label=\arabic*.]
\setcounter{enumi}{10}
\item Mendesain dan mengimplementasikan Airflow DAGs untuk scheduled data processing.
\item Membangun DAG untuk daily ML model retraining dengan dependency management.
\item Mengimplementasikan Airflow sensors untuk monitoring external data sources.
\item Membangun custom Airflow operators untuk specific business logic.
\item Mengkonfigurasi Airflow task retries, timeouts, dan alerting.
\item Mengoptimasi Airflow scheduler performance dan resource utilization.
\item Mengimplementasikan Airflow RBAC untuk access control.
\item Membangun monitoring dashboards untuk Airflow task execution.
\item Mengelola Airflow variable dan connection management securely.
\item Mengimplementasikan CI/CD untuk Airflow DAG deployments.
\end{enumerate}

\subsubsection{Real-time Data Streaming}

\begin{enumerate}[label=\arabic*.]
\setcounter{enumi}{20}
\item Mendesain dan mengimplementasikan Kafka topics untuk real-time scan events.
\item Membangun Kafka producers untuk streaming vulnerability discoveries.
\item Mengimplementasikan Kafka consumers untuk real-time alerting.
\item Membangun stream processing dengan Apache Flink atau Kafka Streams.
\item Mengimplementasikan exactly-once semantics untuk critical data streams.
\item Membangun schema registry untuk Kafka message validation.
\item Mengoptimasi Kafka partition strategy untuk high throughput.
\item Mengimplementasikan stream windowing untuk aggregation analytics.
\item Membangun monitoring untuk stream lag dan throughput metrics.
\item Mengimplementasikan disaster recovery untuk streaming infrastructure.
\end{enumerate}

\subsubsection{Data Versioning dan Lineage}

\begin{enumerate}[label=\arabic*.]
\setcounter{enumi}{30}
\item Mengimplementasikan DVC (Data Version Control) untuk training datasets.
\item Membangun data lineage tracking untuk compliance dan debugging.
\item Mengimplementasikan dataset versioning dengan metadata management.
\item Membangun tools untuk comparing dataset versions.
\item Mengimplementasikan automated data validation pada version changes.
\item Membangun documentation untuk dataset schemas dan changes.
\item Mengimplementasikan rollback mechanisms untuk data versions.
\item Membangun CI/CD integration untuk data pipeline versioning.
\item Mengimplementasikan data quality metrics tracking across versions.
\item Membangun alerting untuk data drift detection.
\end{enumerate}

\subsubsection{Data Lake dan Warehouse Management}

\begin{enumerate}[label=\arabic*.]
\setcounter{enumi}{40}
\item Merancang dan mengimplementasikan data lake architecture.
\item Membangun data ingestion pipelines ke data lake (S3, Azure Data Lake).
\item Mengimplementasikan data partitioning strategy untuk optimal query performance.
\item Membangun data catalog dengan metadata management.
\item Mengimplementasikan data lake security (encryption, access control).
\item Membangun data warehouse schemas (star/snowflake) untuk analytics.
\item Mengimplementasikan incremental loading ke data warehouse.
\item Membangun aggregate tables untuk performance optimization.
\item Mengimplementasikan data retention policies dan archival strategies.
\item Membangun monitoring untuk data lake storage costs dan usage.
\end{enumerate}

\subsubsection{Data Quality dan Validation}

\begin{enumerate}[label=\arabic*.]
\setcounter{enumi}{50}
\item Mengimplementasikan data quality frameworks (Great Expectations, Deequ).
\item Membangun automated data quality tests untuk all pipelines.
\item Mengimplementasikan data profiling untuk understanding dataset characteristics.
\item Membangun anomaly detection untuk identifying data quality issues.
\item Mengimplementasikan data quality dashboards dan reporting.
\item Membangun alerting untuk data quality threshold violations.
\item Mengimplementasikan data remediation workflows.
\item Membangun data quality SLAs dan monitoring.
\item Mengimplementasikan data quality documentation dan standards.
\item Membangun automated data quality reporting untuk stakeholders.
\end{enumerate}

\subsubsection{Performance Optimization}

\begin{enumerate}[label=\arabic*.]
\setcounter{enumi}{60}
\item Mengoptimasi data pipeline execution times untuk SLA compliance.
\item Membangun caching strategies untuk frequently accessed datasets.
\item Mengimplementasikan parallel processing untuk large-scale data operations.
\item Mengoptimasi memory usage pada data transformation operations.
\item Membangun resource monitoring untuk data pipeline infrastructure.
\item Mengimplementasikan auto-scaling untuk data processing workloads.
\item Mengoptimasi network transfer untuk large dataset movements.
\item Membangun compression strategies untuk reducing storage costs.
\item Mengimplementasikan query optimization untuk data warehouse operations.
\item Membangun performance benchmarking untuk continuous improvement.
\end{enumerate}

\subsubsection{Documentation dan Best Practices}

\begin{enumerate}[label=\arabic*.]
\setcounter{enumi}{70}
\item Membuat comprehensive documentation untuk all data pipelines.
\item Membangun data dictionary untuk all datasets dan fields.
\item Mengimplementasikan code review practices untuk data pipeline code.
\item Membangun testing frameworks untuk data pipeline validation.
\item Membuat runbooks untuk common operational tasks.
\item Membangun training materials untuk new team members.
\item Mengimplementasikan version control best practices.
\item Membangun style guides untuk data engineering code.
\item Membuat architecture decision records (ADRs).
\item Membangun knowledge base untuk troubleshooting common issues.
\end{enumerate}

\subsubsection{Integration dan Collaboration}

\begin{enumerate}[label=\arabic*.]
\setcounter{enumi}{80}
\item Berkolaborasi dengan Data Scientists untuk ML pipeline requirements.
\item Berkolaborasi dengan Backend Engineers untuk API data requirements.
\item Berkoordinasi dengan DevOps untuk infrastructure provisioning.
\item Membangun self-service data access untuk analytics team.
\item Mengimplementasikan data sharing mechanisms antar teams.
\item Membangun APIs untuk accessing processed datasets.
\item Berpartisipasi dalam sprint planning dan requirement gathering.
\item Membangun feedback loops dengan data consumers.
\item Mengimplementasikan data governance policies.
\item Membangun cross-functional data quality initiatives.
\end{enumerate}

\subsection{Tanggung Jawab Database Administration (160 Tugas)}

\textit{[Konten Database Administration 160 tugas - sama dengan sebelumnya namun dengan terminologi bootstrap]}

\subsection{Tanggung Jawab DevOps Operations (110 Tugas)}

\textit{[Konten DevOps Operations 110 tugas - sama dengan sebelumnya namun dengan terminologi bootstrap]}

\subsection{Tanggung Jawab Data Management (40 Tugas)}

\textit{[Konten Data Management 40 tugas - sama dengan sebelumnya namun dengan terminologi bootstrap]}

\section{WAKTU KOLABORASI}

\subsection{Jam Kolaborasi}

\begin{enumerate}[label=\alph*)]
\item Hari Kolaborasi: Senin sampai Jumat
\item Jam Kolaborasi: 09:00 - 18:00 WIB (dengan istirahat 1 jam)
\item Total: 40 jam per minggu
\item Remote Work: Diizinkan dengan koordinasi (hybrid model)
\end{enumerate}

\subsection{On-Call Duty}

\begin{enumerate}[label=\alph*)]
\item BOOTSTRAP MEMBER akan tergabung dalam on-call rotation untuk menangani production incidents.
\item On-call duty dilakukan secara bergiliran dengan kompensasi tambahan.
\item Response time untuk P0/P1 incidents: maksimal 15 menit.
\item Escalation procedures akan diatur dalam operational runbook.
\end{enumerate}

\section{KOMPENSASI DAN BENEFIT}

\subsection{Kompensasi Bootstrap}

\begin{enumerate}[label=\alph*)]
\item Equity/Saham: [Jumlah persen] dari total ekuitas perusahaan
\item Allowance Bulanan: Rp [Jumlah] per bulan untuk operasional
\item Revenue Share: [Persen] dari revenue setelah produk launch
\item Bonus Launch: Bonus khusus saat produk berhasil launching
\item Pembayaran allowance dilakukan setiap tanggal 25 setiap bulan
\end{enumerate}

\subsection{Benefit}

\begin{enumerate}[label=\alph*)]
\item Budget untuk training dan sertifikasi profesional (AWS, Kubernetes, PostgreSQL)
\item Laptop dan peralatan yang diperlukan
\item Budget untuk learning platforms dan online courses
\item Internet allowance
\item Flexible working arrangement
\end{enumerate}

\section{KEY PERFORMANCE INDICATORS (KPIs)}

BOOTSTRAP MEMBER akan dievaluasi berdasarkan kontribusi dan pencapaian KPIs berikut:

\subsection{Data Engineering KPIs}

\begin{enumerate}[label=\alph*)]
\item ETL pipeline reliability: $>$ 99.5 persen success rate
\item Data pipeline latency: $<$ 5 menit untuk batch processing
\item Airflow DAG success rate: $>$ 99 persen
\item Data quality score: $>$ 95 persen
\item Streaming pipeline lag: $<$ 100ms
\end{enumerate}

\subsection{Database Administration KPIs}

\begin{enumerate}[label=\alph*)]
\item Database uptime: $>$ 99.9 persen
\item Query performance: P95 $<$ 100ms untuk API queries
\item Backup success rate: 100 persen
\item Recovery Point Objective (RPO): $<$ 1 jam
\item Recovery Time Objective (RTO): $<$ 4 jam
\item Cache hit rate: $>$ 70 persen
\item Test coverage untuk database operations: $>$ 80 persen
\end{enumerate}

\subsection{DevOps KPIs}

\begin{enumerate}[label=\alph*)]
\item System uptime: $>$ 99.9 persen
\item Deployment frequency: Minimal 1x per hari (CD capability)
\item Deployment success rate: $>$ 95 persen
\item Mean Time to Recovery (MTTR): $<$ 30 menit
\item Incident response time: $<$ 15 menit untuk P0/P1
\item Infrastructure cost optimization: 40-60 persen reduction
\item Security vulnerabilities: $<$ 5 high/critical open issues
\end{enumerate}

\subsection{Data Management KPIs}

\begin{enumerate}[label=\alph*)]
\item Data ingestion success rate: $>$ 99 persen
\item Data quality score: $>$ 95 persen
\item Data storage optimization: $<$ 10 persen wasted storage
\item GDPR compliance: 100 persen compliant
\end{enumerate}

\section{KODE ETIK DAN KERAHASIAAN}

\subsection{Kode Etik}

BOOTSTRAP MEMBER wajib:

\begin{enumerate}[label=\alph*)]
\item Mematuhi semua kebijakan dan prosedur bootstrap
\item Menjaga profesionalisme dalam berkolaborasi
\item Berkolaborasi dengan team members secara efektif
\item Melaporkan security incidents atau vulnerabilities segera
\item Mengikuti best practices untuk code quality dan documentation
\item Berpartisipasi dalam code reviews dan knowledge sharing
\end{enumerate}

\subsection{Kerahasiaan}

\begin{enumerate}[label=\alph*)]
\item BOOTSTRAP MEMBER wajib menjaga kerahasiaan semua informasi perusahaan, client data, dan intellectual property.
\item BOOTSTRAP MEMBER dilarang membocorkan vulnerability information atau security details ke pihak luar.
\item Kewajiban kerahasiaan berlaku selama masa bootstrap dan 2 tahun setelah berakhirnya kolaborasi bootstrap.
\item Pelanggaran kerahasiaan dapat mengakibatkan pemutusan kolaborasi bootstrap dan tuntutan hukum.
\end{enumerate}

\section{PELATIHAN DAN PENGEMBANGAN}

\begin{enumerate}[label=\alph*)]
\item PEMBERI ROLE akan menyediakan akses ke training dan development resources.
\item BOOTSTRAP MEMBER berhak mengajukan training atau certification yang relevan dengan role.
\item Budget training: Sesuai kebutuhan bootstrap
\item Certified training yang disarankan:
\begin{itemize}
\item AWS Certified Solutions Architect atau DevOps Engineer
\item Kubernetes CKA (Certified Kubernetes Administrator)
\item PostgreSQL Certified Professional
\item Apache Airflow Fundamentals
\item Terraform Associate Certification
\end{itemize}
\item BOOTSTRAP MEMBER akan mengikuti onboarding program selama 2 minggu pertama.
\end{enumerate}

\section{PEMUTUSAN KOLABORASI BOOTSTRAP}

\subsection{Penarikan Diri}

\begin{enumerate}[label=\alph*)]
\item BOOTSTRAP MEMBER yang ingin keluar wajib memberikan pemberitahuan tertulis minimal 30 hari sebelumnya.
\item Selama notice period, BOOTSTRAP MEMBER wajib melakukan knowledge transfer dan documentation handover.
\item BOOTSTRAP MEMBER wajib mengembalikan semua peralatan dan menghapus semua data dari personal devices.
\end{enumerate}

\subsection{Pemutusan oleh PEMBERI ROLE}

\begin{enumerate}[label=\alph*)]
\item PEMBERI ROLE dapat memutuskan kolaborasi bootstrap dengan pemberitahuan 30 hari atau kompensasi yang disepakati.
\item Pemutusan tanpa pemberitahuan dapat dilakukan jika BOOTSTRAP MEMBER melakukan pelanggaran serius (security breach, data theft, dll).
\item Settlement akan dilakukan sesuai kesepakatan equity dan revenue share.
\end{enumerate}

\section{LAIN-LAIN}

\begin{enumerate}[label=\alph*)]
\item Perjanjian ini dibuat dalam 2 (dua) rangkap yang masing-masing mempunyai kekuatan hukum yang sama.
\item Perubahan atas perjanjian ini harus dibuat secara tertulis dan disetujui oleh PARA PIHAK.
\item Hal-hal yang belum diatur dalam perjanjian ini akan diselesaikan secara musyawarah.
\item Perjanjian ini tunduk pada hukum Republik Indonesia.
\item Perjanjian ini berlaku efektif sejak tanggal yang tercantum di bawah ini.
\end{enumerate}

\vspace{1cm}

Demikian Perjanjian Bootstrap ini dibuat dan ditandatangani pada tanggal tersebut di atas dalam keadaan sehat jasmani dan rohani tanpa adanya paksaan dari pihak manapun.

\vspace{1.5cm}

\begin{minipage}[t]{0.45\textwidth}
\textbf{PEMBERI ROLE}\\[2.5cm]
\rule{6cm}{0.5pt}\\
Hylmi Rafif Rabbani\\
Chief Executive Officer (CEO)\\
PT. Intelligent Digital Knowledge
\end{minipage}

\newpage

\section*{LAMPIRAN A: DETAIL STRUKTUR KOMPENSASI}

\subsection*{A. KEPEMILIKAN SAHAM (EQUITY)}

\begin{enumerate}[label=\arabic*., leftmargin=*]
    \item \textbf{Total Equity}: BOOTSTRAP MEMBER akan menerima kepemilikan saham sebesar \textbf{24,5\%} di PT. Intelligent Digital Knowledge
    
    \item \textbf{Struktur Kepemilikan}:
    \begin{itemize}
        \item Chief Executive Officer (Hylmi Rafif Rabbani): \textbf{51\%}
        \item Abdul Raoef Yudha Qurnia (BOOTSTRAP MEMBER): \textbf{24,5\%}
        \item Bootstrap Member lainnya: \textbf{24,5\%}
    \end{itemize}
    
    \item \textbf{Vesting Schedule} (Jadwal Perolehan Saham):
    \begin{itemize}
        \item Total periode vesting: \textbf{4 (empat) tahun}
        \item \textbf{1-year cliff}: Tidak ada saham yang di-vest dalam 12 bulan pertama
        \item Setelah 12 bulan pertama: \textbf{6,125\%} saham akan di-vest sekaligus
        \item Bulan ke-13 hingga ke-48: Vesting bulanan sebesar \textbf{0,51\%} per bulan
        \item Perhitungan: 24,5\% dibagi 48 bulan = 0,510417\% per bulan
    \end{itemize}
    
    \item \textbf{Contoh Perhitungan Vesting}:
    \begin{itemize}
        \item Bulan 0-11: 0\% vested
        \item Bulan 12: 6,125\% vested (25\% dari 24,5\%)
        \item Bulan 24: 12,25\% vested (50\% dari 24,5\%)
        \item Bulan 36: 18,375\% vested (75\% dari 24,5\%)
        \item Bulan 48: 24,5\% vested (100\% fully vested)
    \end{itemize}
\end{enumerate}

\subsection*{B. TIDAK ADA GAJI BULANAN TETAP}

\begin{enumerate}[label=\arabic*., leftmargin=*]
    \item Selama fase bootstrap, \textbf{TIDAK ADA} gaji bulanan tetap atau tunjangan rutin.
    
    \item Kompensasi berbasis:
    \begin{itemize}
        \item Kepemilikan equity (24,5\%)
        \item Performance bonus berdasarkan pencapaian
        \item Milestone rewards
    \end{itemize}
    
    \item Model ini memastikan \textit{true bootstrap mentality} dan alignment penuh dengan kesuksesan perusahaan.
\end{enumerate}

\subsection*{C. BONUS PERFORMA KUARTALAN}

\textbf{Evaluasi dilakukan setiap 3 (tiga) bulan} dengan kriteria sebagai berikut:

\begin{enumerate}[label=\arabic*., leftmargin=*]
    \item \textbf{TIER EXCEPTIONAL} (Luar Biasa):
    \begin{itemize}
        \item Bonus: \textbf{Rp 15.000.000 - Rp 25.000.000} ATAU \textbf{+0,5\% equity tambahan}
        \item Kriteria:
        \begin{itemize}
            \item Melebihi SEMUA target KPI lebih dari 20\%
            \item Pekerjaan manual $>$50\% (penggunaan AI $<$50\%)
            \item Tidak ada critical bugs atau issues
            \item Mengimplementasikan solusi breakthrough/inovatif
            \item Kolaborasi tim sangat baik
            \item Dokumentasi dan knowledge sharing excellent
        \end{itemize}
    \end{itemize}
    
    \item \textbf{TIER EXCELLENT} (Sangat Baik):
    \begin{itemize}
        \item Bonus: \textbf{Rp 8.000.000 - Rp 15.000.000} ATAU \textbf{+0,25\% equity tambahan}
        \item Kriteria:
        \begin{itemize}
            \item Mencapai SEMUA target KPI + melebihi beberapa target $>$10\%
            \item Pekerjaan manual $>$60\% (penggunaan AI $<$40\%)
            \item Bug minimal, resolusi cepat
            \item Output berkualitas tinggi secara konsisten
            \item Kolaborasi tim baik
        \end{itemize}
    \end{itemize}
    
    \item \textbf{TIER GOOD} (Baik):
    \begin{itemize}
        \item Bonus: \textbf{Rp 3.000.000 - Rp 7.000.000}
        \item Kriteria:
        \begin{itemize}
            \item Mencapai SEMUA target KPI
            \item Pekerjaan manual $>$70\% (penggunaan AI $<$30\%)
            \item Kualitas acceptable
            \item Performa standard
        \end{itemize}
    \end{itemize}
    
    \item \textbf{TIER NEEDS IMPROVEMENT} (Perlu Perbaikan):
    \begin{itemize}
        \item \textbf{Tidak ada bonus}
        \item Kriteria:
        \begin{itemize}
            \item Tidak mencapai beberapa target KPI
            \item Over-reliance pada AI ($>$50\% pekerjaan di-generate AI)
            \item Quality issues
            \item Performance improvement plan akan diberlakukan
        \end{itemize}
    \end{itemize}
\end{enumerate}

\subsection*{D. KEBIJAKAN PENGGUNAAN AI DAN PENALTY/REWARD}

\textbf{Filosofi}: Menghargai skill asli dan usaha, bukan copy-paste dari AI.

\begin{enumerate}[label=\arabic*., leftmargin=*]
    \item \textbf{AI Usage $<$30\%} (Master Craftsman):
    \begin{itemize}
        \item Reward: \textbf{+Rp 5.000.000} bonus kuartalan
        \item Pengakuan sebagai "Master Craftsman"
        \item Menunjukkan genuine problem-solving dan deep understanding
    \end{itemize}
    
    \item \textbf{AI Usage 30-50\%} (Balanced):
    \begin{itemize}
        \item Standard (tidak ada penalty, tidak ada bonus tambahan)
        \item Pendekatan seimbang diterima
    \end{itemize}
    
    \item \textbf{AI Usage 50-70\%} (Warning Zone):
    \begin{itemize}
        \item Penalty: \textbf{-Rp 3.000.000} dari bonus kuartalan
        \item Warning formal: Perlu improve manual skills
        \item Performance monitoring ketat
    \end{itemize}
    
    \item \textbf{AI Usage $>$70\%} (Critical):
    \begin{itemize}
        \item Penalty: \textbf{-Rp 10.000.000} dari bonus kuartalan
        \item Performance review wajib dilakukan
        \item Risiko: Equity vesting dapat di-pause
        \item Dapat menjadi alasan pemberhentian jika tidak ada perbaikan
    \end{itemize}
\end{enumerate}

\textbf{Cara Pengukuran AI Usage}:
\begin{itemize}
    \item Code review dan assessment kualitas
    \item Evaluasi pendekatan problem-solving
    \item Demonstrasi debugging capability
    \item Justifikasi keputusan arsitektur
    \item Original thinking vs pattern copying
    \item Technical interview periodik
\end{itemize}

\subsection*{E. BONUS MILESTONE (One-time Cash/Equity)}

\textbf{Product Milestones}:
\begin{itemize}
    \item MVP Launch (production-ready): \textbf{Rp 10.000.000}
    \item 1.000 Active Users: \textbf{Rp 15.000.000}
    \item 10.000 Active Users: \textbf{Rp 30.000.000}
    \item Revenue pertama Rp 100 juta: \textbf{Rp 50.000.000}
    \item Product-Market Fit tercapai: \textbf{Rp 100.000.000}
\end{itemize}

\textbf{Funding Milestones}:
\begin{itemize}
    \item Pre-seed (Rp 1-5 Miliar): \textbf{+1\% equity tambahan}
    \item Seed (Rp 5-20 Miliar): \textbf{+2\% equity tambahan}
    \item Series A (Rp 20 Miliar+): \textbf{+3\% equity tambahan}
\end{itemize}

\subsection*{F. SPECIAL ACHIEVEMENT AWARDS (Ad-hoc)}

\textbf{Innovation Award}:
\begin{itemize}
    \item Implementasi game-changing feature: \textbf{Rp 20.000.000}
    \item Menyelesaikan critical architectural issue: \textbf{Rp 15.000.000}
    \item Patent-worthy innovation: \textbf{Rp 50.000.000 + 1\% equity}
\end{itemize}

\textbf{Quality Award}:
\begin{itemize}
    \item Zero bugs selama 6 bulan: \textbf{Rp 10.000.000}
    \item Maintain 95\%+ test coverage: \textbf{Rp 5.000.000}
    \item Menemukan security vulnerability: \textbf{Rp 15.000.000}
\end{itemize}

\textbf{Growth Award}:
\begin{itemize}
    \item Feature mendorong 50\% user growth: \textbf{Rp 25.000.000}
    \item Infrastructure menghemat $>$30\% cost: \textbf{Rp 20.000.000}
    \item Partnership bernilai $>$Rp 100 juta: \textbf{Rp 30.000.000}
\end{itemize}

\subsection*{G. KOMPENSASI POST-FUNDING/REVENUE STABIL}

\textbf{Setelah perusahaan mencapai salah satu kondisi berikut}:
\begin{itemize}
    \item Revenue stabil $>$Rp 500 juta per bulan, ATAU
    \item Mendapat funding Series A atau lebih tinggi
\end{itemize}

\textbf{Maka berlaku}:
\begin{enumerate}[label=\arabic*., leftmargin=*]
    \item Gaji tetap bulanan: \textbf{Rp 15.000.000 - Rp 25.000.000} (disesuaikan dengan market rate)
    \item Equity vesting tetap berlanjut sesuai jadwal
    \item Performance bonus kuartalan tetap berlaku
    \item Benefits tambahan:
    \begin{itemize}
        \item BPJS Kesehatan dan Ketenagakerjaan
        \item Asuransi kesehatan swasta
        \item Annual leave: 12 hari per tahun
        \item Training budget: Rp 10.000.000 per tahun
    \end{itemize}
\end{enumerate}

\subsection*{H. CATATAN PENTING}

\begin{enumerate}[label=\arabic*., leftmargin=*]
    \item Semua bonus dan reward dalam bentuk cash akan dibayarkan dalam Rupiah melalui transfer bank.
    \item Bonus equity tambahan akan mengikuti vesting schedule yang sama (4 tahun, 1-year cliff).
    \item Evaluasi performa dilakukan secara transparan dengan dokumentasi lengkap.
    \item BOOTSTRAP MEMBER berhak meminta klarifikasi dan feedback atas evaluasi performa.
    \item Dalam hal dispute mengenai performa atau AI usage assessment, akan diselesaikan melalui diskusi dan review bersama.
\end{enumerate}

\vspace{1.5cm}

\noindent\textit{Lampiran ini merupakan bagian yang tidak terpisahkan dari Perjanjian Bootstrap Nomor PB/IDK/2025/001.}

\newpage

\section*{HALAMAN PENANDATANGANAN}

\vspace{1cm}

\noindent Demikian Perjanjian Bootstrap ini dibuat dengan sebenarnya dalam keadaan sadar, tanpa paksaan dari pihak manapun, dan ditandatangani oleh PARA PIHAK:

\vspace{2cm}

\begin{minipage}[t]{0.45\textwidth}
\textbf{PEMBERI ROLE}\\[2.5cm]
\rule{6cm}{0.5pt}\\
Hylmi Rafif Rabbani\\
Chief Executive Officer (CEO)\\
PT. Intelligent Digital Knowledge
\end{minipage}
\hfill
\begin{minipage}[t]{0.45\textwidth}
\textbf{BOOTSTRAP MEMBER}\\[2.5cm]
\rule{6cm}{0.5pt}\\
Abdul Raoef Yudha Qurnia\\
Data Engineer, Database Administrator,\\
dan DevOps Engineer
\end{minipage}

\vspace{1.5cm}

\noindent\textit{Materai Rp 10.000}

\end{document}
