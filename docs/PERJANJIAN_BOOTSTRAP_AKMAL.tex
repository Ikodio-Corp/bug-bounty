\documentclass[12pt,a4paper]{article}
\usepackage[utf8]{inputenc}
\usepackage[indonesian]{babel}
\usepackage{geometry}
\usepackage{enumitem}
\usepackage{titlesec}
\usepackage{fancyhdr}
\usepackage{setspace}

\geometry{a4paper, margin=2.5cm}
\setstretch{1.15}
\pagestyle{fancy}
\fancyhf{}
\fancyhead[L]{\small PERJANJIAN BOOTSTRAP}
\fancyhead[R]{\small Halaman \thepage}
\renewcommand{\headrulewidth}{0.4pt}

\titleformat{\section}{\normalfont\Large\bfseries}{Pasal \thesection.}{0.5em}{}
\titleformat{\subsection}{\normalfont\large\bfseries}{\thesubsection.}{0.5em}{}

\begin{document}

\begin{center}
    \textbf{\LARGE PERJANJIAN BOOTSTRAP}\\[0.5cm]
    \textbf{Nomor: PB/IDK/2025/002}\\[1cm]
\end{center}

\noindent Pada hari ini, \textbf{[Tanggal]}, telah dibuat dan ditandatangani Perjanjian Bootstrap ini oleh dan antara:

\begin{enumerate}[label=\arabic*., leftmargin=*]
    \item \textbf{PEMBERI ROLE}
    \begin{itemize}[label=, leftmargin=0pt]
        \item Nama: \textbf{Hylmi Rafif Rabbani}
        \item Jabatan: Chief Executive Officer (CEO)
        \item Perusahaan: \textbf{PT. Intelligent Digital Knowledge}
        \item Alamat: [Alamat Perusahaan]
        
        \vspace{0.3cm}
        Selanjutnya disebut sebagai \textbf{PEMBERI ROLE}
    \end{itemize}
    
    \item \textbf{BOOTSTRAP MEMBER}
    \begin{itemize}[label=, leftmargin=0pt]
        \item Nama: \textbf{Hafizhul Akmal}
        \item NIK: [Nomor Induk Kependudukan]
        \item Alamat: [Alamat Lengkap]
        \item No. HP: [Nomor HP]
        \item Email: [Email]
        
        \vspace{0.3cm}
        Selanjutnya disebut sebagai \textbf{BOOTSTRAP MEMBER}
    \end{itemize}
\end{enumerate}

\vspace{0.5cm}
\noindent PEMBERI ROLE dan BOOTSTRAP MEMBER secara bersama-sama disebut sebagai \textbf{PARA PIHAK}, dengan ini sepakat untuk mengadakan Perjanjian Bootstrap dengan ketentuan dan syarat-syarat sebagai berikut:

\section{PENGANGKATAN DAN JABATAN}

\begin{enumerate}[label=\alph*), leftmargin=*]
    \item PEMBERI ROLE dengan ini mengikutsertakan BOOTSTRAP MEMBER untuk berkolaborasi di PT. Intelligent Digital Knowledge dalam kapasitas bootstrap phase.
    
    \item BOOTSTRAP MEMBER ditempatkan pada posisi:
    \begin{itemize}
        \item \textbf{Full Stack Developer} (Next.js + FastAPI)
        \item \textbf{Data Analyst \& Marketing Specialist}
        \item \textbf{ML UI/UX Developer}
    \end{itemize}
    
    \item Status: \textbf{Bootstrap Member} (masa bootstrap hingga produk launch dan mendapat funding/revenue stabil)
    
    \item BOOTSTRAP MEMBER akan berkolaborasi langsung dengan CEO dan tim dalam mengembangkan produk dari fase awal hingga market-ready.
\end{enumerate}

\section{LINGKUP ROLE DAN TANGGUNG JAWAB}

BOOTSTRAP MEMBER bertanggung jawab penuh atas role dan tugas-tugas berikut:

\subsection{Full Stack Development (Next.js + FastAPI)}

\textbf{Backend Development (FastAPI)}
\begin{enumerate}[label=\arabic*., leftmargin=*]
    \item Mengembangkan RESTful API menggunakan FastAPI untuk semua fitur aplikasi
    \item Implementasi endpoint CRUD untuk manajemen bugs, scans, users, guilds, marketplace
    \item Integrasi dengan PostgreSQL database melalui SQLAlchemy ORM
    \item Implementasi authentication dan authorization (JWT, OAuth2)
    \item Mengembangkan middleware untuk logging, error handling, dan rate limiting
    \item Integrasi dengan Redis untuk caching dan session management
    \item Implementasi WebSocket untuk real-time notifications dan updates
    \item Integrasi dengan external services (Stripe, email, cloud storage)
    \item Mengembangkan Celery tasks untuk async operations
    \item Implementasi API documentation menggunakan Swagger/OpenAPI
    \item Testing backend dengan pytest (unit, integration tests)
    \item Performance optimization dan query optimization
    \item Database migration management dengan Alembic
    \item Implementation of AI agents integration endpoints
    \item Monitoring dan logging backend operations
\end{enumerate}

\textbf{Frontend Development (Next.js)}
\begin{enumerate}[label=\arabic*., leftmargin=*, resume]
    \item Mengembangkan UI components menggunakan React dan Next.js 14+
    \item Implementasi App Router dan server components
    \item State management dengan React Context/Zustand/Redux
    \item Integrasi dengan backend API menggunakan fetch/axios
    \item Implementasi authentication flow di frontend
    \item Responsive design untuk mobile, tablet, dan desktop
    \item Styling dengan Tailwind CSS
    \item Implementasi dark/light mode
    \item SEO optimization dengan Next.js metadata
    \item Performance optimization (lazy loading, code splitting)
    \item Implementasi forms dengan validation (React Hook Form, Zod)
    \item Real-time updates dengan WebSocket integration
    \item Testing frontend dengan Jest dan React Testing Library
    \item Accessibility (a11y) compliance
    \item Build optimization dan deployment configuration
\end{enumerate}

\subsection{Data Analysis \& Business Intelligence}

\begin{enumerate}[label=\arabic*., leftmargin=*]
    \item Menganalisis data bug reports untuk mendapatkan insights
    \item Membuat dashboard visualisasi untuk metrics dan KPIs
    \item Analisis user behavior dan engagement patterns
    \item Conversion funnel analysis untuk marketplace
    \item Cohort analysis untuk user retention
    \item A/B testing analysis untuk features
    \item SQL query optimization untuk analytics
    \item Membuat automated reports dengan Python (Pandas, NumPy)
    \item Data visualization dengan Plotly, Matplotlib, atau D3.js
    \item Integration dengan BI tools (Grafana, Metabase)
    \item Predictive analytics untuk bug severity dan impact
    \item Market analysis dan competitor research
    \item Revenue analytics dan financial reporting
    \item Customer segmentation analysis
    \item Performance metrics tracking dan reporting
\end{enumerate}

\subsection{Digital Marketing \& Growth}

\textbf{Growth Marketing}
\begin{enumerate}[label=\arabic*., leftmargin=*]
    \item Merancang dan melaksanakan strategi growth hacking
    \item User acquisition campaigns (organic dan paid)
    \item Conversion rate optimization (CRO)
    \item Referral program development dan management
    \item Partnership dan collaboration strategies
    \item Community building dan engagement
    \item Product-led growth initiatives
    \item Viral loop optimization
\end{enumerate}

\textbf{Content Marketing}
\begin{enumerate}[label=\arabic*., leftmargin=*, resume]
    \item Membuat content strategy dan editorial calendar
    \item Blog posts, tutorials, dan technical articles
    \item Case studies dan success stories
    \item Video content creation dan editing
    \item Infographics dan visual content
    \item Newsletter creation dan management
    \item Documentation dan knowledge base content
    \item Guest posting dan content syndication
\end{enumerate}

\textbf{SEO \& SEM}
\begin{enumerate}[label=\arabic*., leftmargin=*, resume]
    \item On-page SEO optimization
    \item Technical SEO implementation
    \item Keyword research dan competitive analysis
    \item Backlink building strategy
    \item Local SEO (jika applicable)
    \item Google Ads campaign management
    \item Performance tracking dengan Google Analytics dan Search Console
    \item Conversion tracking dan attribution modeling
\end{enumerate}

\textbf{Social Media Marketing}
\begin{enumerate}[label=\arabic*., leftmargin=*, resume]
    \item Social media strategy development
    \item Content creation untuk Twitter, LinkedIn, Instagram, Facebook
    \item Community management dan engagement
    \item Social media advertising campaigns
    \item Influencer collaboration
    \item Social listening dan reputation management
    \item Analytics dan reporting
    \item Trend monitoring dan quick response content
\end{enumerate}

\subsection{ML UI/UX Development}

\begin{enumerate}[label=\arabic*., leftmargin=*]
    \item Desain UI untuk ML prediction interfaces
    \item Implementasi interactive ML model visualization
    \item Real-time prediction result display
    \item Model explainability UI (SHAP, LIME visualizations)
    \item Confidence score indicators dan uncertainty visualization
    \item User-friendly input forms untuk ML features
    \item Automated suggestions berdasarkan ML predictions
    \item A/B testing interface untuk ML model versions
    \item Performance metrics dashboard untuk ML models
    \item User feedback collection untuk model improvement
    \item Error handling dan fallback UI untuk ML failures
    \item Progressive enhancement untuk ML features
    \item Responsive ML interfaces untuk mobile devices
    \item Accessibility considerations untuk ML UI
    \item Integration dengan MLOps pipeline
\end{enumerate}

\section{WAKTU KOLABORASI BOOTSTRAP}

\begin{enumerate}[label=\alph*), leftmargin=*]
    \item Kolaborasi bootstrap dimulai pada tanggal \textbf{[Tanggal Mulai]}.
    
    \item Tidak ada batas waktu yang kaku; kolaborasi berlanjut hingga:
    \begin{itemize}
        \item Produk mencapai product-market fit
        \item Mendapat funding dari investor, atau
        \item Mencapai revenue yang stabil dan berkelanjutan
    \end{itemize}
    
    \item Jam kerja: Fleksibel dengan target deliverables yang clear, dengan ekspektasi minimum 40 jam/minggu untuk sprint-based development.
    
    \item Remote/hybrid working arrangement dapat diatur sesuai kebutuhan tim dan project phase.
\end{enumerate}

\section{KOMPENSASI DAN BENEFIT}

\begin{enumerate}[label=\alph*), leftmargin=*]
    \item \textbf{Fase Bootstrap (Pre-Funding/Revenue)}:
    \begin{itemize}
        \item Equity/profit sharing: \textbf{[X]\%} dari company valuation atau future profit
        \item Token allowance untuk operasional: \textbf{Rp [Jumlah]} per bulan
        \item Kompensasi penuh akan dibayar ketika funding diperoleh atau revenue stabil tercapai
    \end{itemize}
    
    \item \textbf{Post-Funding/Revenue}:
    \begin{itemize}
        \item Gaji tetap: \textbf{Rp [Jumlah]} per bulan
        \item Equity tetap: \textbf{[X]\%} vested over [Y] years
        \item Performance bonus berdasarkan KPI dan company growth
        \item Health insurance (BPJS Kesehatan + private insurance)
        \item Annual leave: 12 hari per tahun
        \item Training budget: Rp [Jumlah] per tahun
    \end{itemize}
    
    \item Semua tools, software licenses, dan infrastruktur development disediakan perusahaan.
\end{enumerate}

\section{KEY PERFORMANCE INDICATORS (KPIs)}

BOOTSTRAP MEMBER akan dievaluasi berdasarkan KPIs berikut:

\subsection{Development KPIs}
\begin{itemize}
    \item Feature completion rate (sprint velocity)
    \item Code quality metrics (test coverage, code review approval rate)
    \item Bug fix turnaround time
    \item API response time dan performance metrics
    \item Frontend performance (Core Web Vitals)
    \item Deployment frequency dan success rate
\end{itemize}

\subsection{Data \& Analytics KPIs}
\begin{itemize}
    \item Quality dan timeliness of insights delivered
    \item Dashboard adoption rate by stakeholders
    \item Data accuracy dan reliability
    \item Report automation coverage
    \item Impact of insights on business decisions
\end{itemize}

\subsection{Marketing KPIs}
\begin{itemize}
    \item Monthly Active Users (MAU) growth
    \item User acquisition cost (CAC)
    \item Conversion rate improvements
    \item Social media engagement rate
    \item Content reach dan impressions
    \item SEO rankings untuk target keywords
    \item Email open rate dan click-through rate
    \item Marketing ROI
\end{itemize}

\subsection{ML UI KPIs}
\begin{itemize}
    \item ML feature adoption rate
    \item User satisfaction dengan ML predictions
    \item ML interface usability scores
    \item Prediction accuracy visualization clarity
    \item User feedback collection rate
\end{itemize}

\section{KODE ETIK DAN KERAHASIAAN}

\begin{enumerate}[label=\alph*), leftmargin=*]
    \item BOOTSTRAP MEMBER wajib menjaga kerahasiaan informasi perusahaan, termasuk:
    \begin{itemize}
        \item Source code dan intellectual property
        \item Data pelanggan dan user data
        \item Strategi bisnis dan marketing plans
        \item Financial information
        \item Trade secrets dan proprietary algorithms
    \end{itemize}
    
    \item Non-Disclosure Agreement (NDA) berlaku selama masa kolaborasi dan 2 tahun setelahnya.
    
    \item Non-Compete: Selama masa kolaborasi, BOOTSTRAP MEMBER tidak boleh terlibat dalam proyek kompetitor langsung.
    
    \item Intellectual Property: Semua hasil kerja (code, designs, content) menjadi milik PT. Intelligent Digital Knowledge.
    
    \item BOOTSTRAP MEMBER wajib mengikuti:
    \begin{itemize}
        \item Code of conduct perusahaan
        \item Security best practices
        \item Data protection regulations (GDPR, Indonesia's UU PDP)
        \item Professional ethics dalam komunikasi dengan stakeholders
    \end{itemize}
\end{enumerate}

\section{PELATIHAN DAN PENGEMBANGAN}

\begin{enumerate}[label=\alph*), leftmargin=*]
    \item PEMBERI ROLE akan menyediakan akses ke training resources:
    \begin{itemize}
        \item Online courses (Udemy, Coursera, Pluralsight, etc.)
        \item Technical books dan documentation
        \item Conference tickets (jika applicable)
        \item Mentorship dari senior developers
    \end{itemize}
    
    \item BOOTSTRAP MEMBER diharapkan untuk continuous learning:
    \begin{itemize}
        \item Stay updated dengan latest tech trends
        \item Share knowledge dengan tim (internal tech talks)
        \item Contribute to technical documentation
    \end{itemize}
\end{enumerate}

\section{PEMUTUSAN KOLABORASI BOOTSTRAP}

\begin{enumerate}[label=\alph*), leftmargin=*]
    \item Kolaborasi dapat diakhiri oleh salah satu pihak dengan pemberitahuan tertulis:
    \begin{itemize}
        \item 30 hari sebelumnya untuk pengunduran diri
        \item Immediate termination untuk pelanggaran berat (dengan bukti)
    \end{itemize}
    
    \item Alasan pemutusan yang dapat diterima:
    \begin{itemize}
        \item Underperformance berkepanjangan (setelah performance improvement plan)
        \item Pelanggaran NDA atau code of conduct
        \item Fraud atau misconduct
        \item Mutual agreement untuk berpisah
        \item Force majeure
    \end{itemize}
    
    \item Hak dan kewajiban saat pemutusan:
    \begin{itemize}
        \item Equity yang sudah vested tetap menjadi hak BOOTSTRAP MEMBER
        \item Kompensasi yang tertunggak akan dibayar sesuai agreement
        \item Handover semua work-in-progress dan documentation
        \item Return semua company assets (laptop, access keys, etc.)
    \end{itemize}
\end{enumerate}

\section{LAIN-LAIN}

\begin{enumerate}[label=\alph*), leftmargin=*]
    \item Perjanjian ini dibuat dalam rangkap 2 (dua), masing-masing bermaterai cukup dan mempunyai kekuatan hukum yang sama.
    
    \item Segala perubahan atau penambahan terhadap Perjanjian ini harus dibuat secara tertulis dan ditandatangani oleh PARA PIHAK.
    
    \item Dalam hal terjadi perselisihan, PARA PIHAK sepakat untuk menyelesaikannya secara musyawarah. Jika tidak tercapai kesepakatan, maka akan diselesaikan melalui jalur hukum yang berlaku di Indonesia.
    
    \item Perjanjian ini tunduk pada hukum Negara Republik Indonesia.
\end{enumerate}

\newpage

\section*{LAMPIRAN A: DETAIL STRUKTUR KOMPENSASI}

\subsection*{A. KEPEMILIKAN SAHAM (EQUITY)}

\begin{enumerate}[label=\arabic*., leftmargin=*]
    \item \textbf{Total Equity}: BOOTSTRAP MEMBER akan menerima kepemilikan saham sebesar \textbf{24,5\%} di PT. Intelligent Digital Knowledge
    
    \item \textbf{Struktur Kepemilikan}:
    \begin{itemize}
        \item Chief Executive Officer (Hylmi Rafif Rabbani): \textbf{51\%}
        \item Hafizhul Akmal (BOOTSTRAP MEMBER): \textbf{24,5\%}
        \item Bootstrap Member lainnya: \textbf{24,5\%}
    \end{itemize}
    
    \item \textbf{Vesting Schedule} (Jadwal Perolehan Saham):
    \begin{itemize}
        \item Total periode vesting: \textbf{4 (empat) tahun}
        \item \textbf{1-year cliff}: Tidak ada saham yang di-vest dalam 12 bulan pertama
        \item Setelah 12 bulan pertama: \textbf{6,125\%} saham akan di-vest sekaligus
        \item Bulan ke-13 hingga ke-48: Vesting bulanan sebesar \textbf{0,51\%} per bulan
        \item Perhitungan: 24,5\% dibagi 48 bulan = 0,510417\% per bulan
    \end{itemize}
    
    \item \textbf{Contoh Perhitungan Vesting}:
    \begin{itemize}
        \item Bulan 0-11: 0\% vested
        \item Bulan 12: 6,125\% vested (25\% dari 24,5\%)
        \item Bulan 24: 12,25\% vested (50\% dari 24,5\%)
        \item Bulan 36: 18,375\% vested (75\% dari 24,5\%)
        \item Bulan 48: 24,5\% vested (100\% fully vested)
    \end{itemize}
\end{enumerate}

\subsection*{B. TIDAK ADA GAJI BULANAN TETAP}

\begin{enumerate}[label=\arabic*., leftmargin=*]
    \item Selama fase bootstrap, \textbf{TIDAK ADA} gaji bulanan tetap atau tunjangan rutin.
    
    \item Kompensasi berbasis:
    \begin{itemize}
        \item Kepemilikan equity (24,5\%)
        \item Performance bonus berdasarkan pencapaian
        \item Milestone rewards
    \end{itemize}
    
    \item Model ini memastikan \textit{true bootstrap mentality} dan alignment penuh dengan kesuksesan perusahaan.
\end{enumerate}

\subsection*{C. BONUS PERFORMA KUARTALAN}

\textbf{Evaluasi dilakukan setiap 3 (tiga) bulan} dengan kriteria sebagai berikut:

\begin{enumerate}[label=\arabic*., leftmargin=*]
    \item \textbf{TIER EXCEPTIONAL} (Luar Biasa):
    \begin{itemize}
        \item Bonus: \textbf{Rp 15.000.000 - Rp 25.000.000} ATAU \textbf{+0,5\% equity tambahan}
        \item Kriteria:
        \begin{itemize}
            \item Melebihi SEMUA target KPI lebih dari 20\%
            \item Pekerjaan manual $>$50\% (penggunaan AI $<$50\%)
            \item Tidak ada critical bugs atau issues
            \item Mengimplementasikan solusi breakthrough/inovatif
            \item Kolaborasi tim sangat baik
            \item Dokumentasi dan knowledge sharing excellent
        \end{itemize}
    \end{itemize}
    
    \item \textbf{TIER EXCELLENT} (Sangat Baik):
    \begin{itemize}
        \item Bonus: \textbf{Rp 8.000.000 - Rp 15.000.000} ATAU \textbf{+0,25\% equity tambahan}
        \item Kriteria:
        \begin{itemize}
            \item Mencapai SEMUA target KPI + melebihi beberapa target $>$10\%
            \item Pekerjaan manual $>$60\% (penggunaan AI $<$40\%)
            \item Bug minimal, resolusi cepat
            \item Output berkualitas tinggi secara konsisten
            \item Kolaborasi tim baik
        \end{itemize}
    \end{itemize}
    
    \item \textbf{TIER GOOD} (Baik):
    \begin{itemize}
        \item Bonus: \textbf{Rp 3.000.000 - Rp 7.000.000}
        \item Kriteria:
        \begin{itemize}
            \item Mencapai SEMUA target KPI
            \item Pekerjaan manual $>$70\% (penggunaan AI $<$30\%)
            \item Kualitas acceptable
            \item Performa standard
        \end{itemize}
    \end{itemize}
    
    \item \textbf{TIER NEEDS IMPROVEMENT} (Perlu Perbaikan):
    \begin{itemize}
        \item \textbf{Tidak ada bonus}
        \item Kriteria:
        \begin{itemize}
            \item Tidak mencapai beberapa target KPI
            \item Over-reliance pada AI ($>$50\% pekerjaan di-generate AI)
            \item Quality issues
            \item Performance improvement plan akan diberlakukan
        \end{itemize}
    \end{itemize}
\end{enumerate}

\subsection*{D. KEBIJAKAN PENGGUNAAN AI DAN PENALTY/REWARD}

\textbf{Filosofi}: Menghargai skill asli dan usaha, bukan copy-paste dari AI.

\begin{enumerate}[label=\arabic*., leftmargin=*]
    \item \textbf{AI Usage $<$30\%} (Master Craftsman):
    \begin{itemize}
        \item Reward: \textbf{+Rp 5.000.000} bonus kuartalan
        \item Pengakuan sebagai "Master Craftsman"
        \item Menunjukkan genuine problem-solving dan deep understanding
    \end{itemize}
    
    \item \textbf{AI Usage 30-50\%} (Balanced):
    \begin{itemize}
        \item Standard (tidak ada penalty, tidak ada bonus tambahan)
        \item Pendekatan seimbang diterima
    \end{itemize}
    
    \item \textbf{AI Usage 50-70\%} (Warning Zone):
    \begin{itemize}
        \item Penalty: \textbf{-Rp 3.000.000} dari bonus kuartalan
        \item Warning formal: Perlu improve manual skills
        \item Performance monitoring ketat
    \end{itemize}
    
    \item \textbf{AI Usage $>$70\%} (Critical):
    \begin{itemize}
        \item Penalty: \textbf{-Rp 10.000.000} dari bonus kuartalan
        \item Performance review wajib dilakukan
        \item Risiko: Equity vesting dapat di-pause
        \item Dapat menjadi alasan pemberhentian jika tidak ada perbaikan
    \end{itemize}
\end{enumerate}

\textbf{Cara Pengukuran AI Usage}:
\begin{itemize}
    \item Code review dan assessment kualitas
    \item Evaluasi pendekatan problem-solving
    \item Demonstrasi debugging capability
    \item Justifikasi keputusan arsitektur
    \item Original thinking vs pattern copying
    \item Technical interview periodik
\end{itemize}

\subsection*{E. BONUS MILESTONE (One-time Cash/Equity)}

\textbf{Product Milestones}:
\begin{itemize}
    \item MVP Launch (production-ready): \textbf{Rp 10.000.000}
    \item 1.000 Active Users: \textbf{Rp 15.000.000}
    \item 10.000 Active Users: \textbf{Rp 30.000.000}
    \item Revenue pertama Rp 100 juta: \textbf{Rp 50.000.000}
    \item Product-Market Fit tercapai: \textbf{Rp 100.000.000}
\end{itemize}

\textbf{Funding Milestones}:
\begin{itemize}
    \item Pre-seed (Rp 1-5 Miliar): \textbf{+1\% equity tambahan}
    \item Seed (Rp 5-20 Miliar): \textbf{+2\% equity tambahan}
    \item Series A (Rp 20 Miliar+): \textbf{+3\% equity tambahan}
\end{itemize}

\subsection*{F. SPECIAL ACHIEVEMENT AWARDS (Ad-hoc)}

\textbf{Innovation Award}:
\begin{itemize}
    \item Implementasi game-changing feature: \textbf{Rp 20.000.000}
    \item Menyelesaikan critical architectural issue: \textbf{Rp 15.000.000}
    \item Patent-worthy innovation: \textbf{Rp 50.000.000 + 1\% equity}
\end{itemize}

\textbf{Quality Award}:
\begin{itemize}
    \item Zero bugs selama 6 bulan: \textbf{Rp 10.000.000}
    \item Maintain 95\%+ test coverage: \textbf{Rp 5.000.000}
    \item Menemukan security vulnerability: \textbf{Rp 15.000.000}
\end{itemize}

\textbf{Growth Award}:
\begin{itemize}
    \item Feature mendorong 50\% user growth: \textbf{Rp 25.000.000}
    \item Marketing campaign dengan ROI $>$10x: \textbf{Rp 20.000.000}
    \item Partnership bernilai $>$Rp 100 juta: \textbf{Rp 30.000.000}
\end{itemize}

\subsection*{G. KOMPENSASI POST-FUNDING/REVENUE STABIL}

\textbf{Setelah perusahaan mencapai salah satu kondisi berikut}:
\begin{itemize}
    \item Revenue stabil $>$Rp 500 juta per bulan, ATAU
    \item Mendapat funding Series A atau lebih tinggi
\end{itemize}

\textbf{Maka berlaku}:
\begin{enumerate}[label=\arabic*., leftmargin=*]
    \item Gaji tetap bulanan: \textbf{Rp 15.000.000 - Rp 25.000.000} (disesuaikan dengan market rate)
    \item Equity vesting tetap berlanjut sesuai jadwal
    \item Performance bonus kuartalan tetap berlaku
    \item Benefits tambahan:
    \begin{itemize}
        \item BPJS Kesehatan dan Ketenagakerjaan
        \item Asuransi kesehatan swasta
        \item Annual leave: 12 hari per tahun
        \item Training budget: Rp 10.000.000 per tahun
    \end{itemize}
\end{enumerate}

\subsection*{H. CATATAN PENTING}

\begin{enumerate}[label=\arabic*., leftmargin=*]
    \item Semua bonus dan reward dalam bentuk cash akan dibayarkan dalam Rupiah melalui transfer bank.
    \item Bonus equity tambahan akan mengikuti vesting schedule yang sama (4 tahun, 1-year cliff).
    \item Evaluasi performa dilakukan secara transparan dengan dokumentasi lengkap.
    \item BOOTSTRAP MEMBER berhak meminta klarifikasi dan feedback atas evaluasi performa.
    \item Dalam hal dispute mengenai performa atau AI usage assessment, akan diselesaikan melalui diskusi dan review bersama.
\end{enumerate}

\vspace{1.5cm}

\noindent\textit{Lampiran ini merupakan bagian yang tidak terpisahkan dari Perjanjian Bootstrap Nomor PB/IDK/2025/002.}

\newpage

\section*{HALAMAN PENANDATANGANAN}

\vspace{1cm}

\noindent Demikian Perjanjian Bootstrap ini dibuat dengan sebenarnya dalam keadaan sadar, tanpa paksaan dari pihak manapun, dan ditandatangani oleh PARA PIHAK:

\vspace{2cm}

\begin{minipage}[t]{0.45\textwidth}
\textbf{PEMBERI ROLE}\\[2.5cm]
\rule{6cm}{0.5pt}\\
Hylmi Rafif Rabbani\\
Chief Executive Officer (CEO)\\
PT. Intelligent Digital Knowledge
\end{minipage}
\hfill
\begin{minipage}[t]{0.45\textwidth}
\textbf{BOOTSTRAP MEMBER}\\[2.5cm]
\rule{6cm}{0.5pt}\\
Hafizhul Akmal\\
Full Stack Developer,\\
Data Analyst \& Marketing Specialist
\end{minipage}

\vspace{1.5cm}

\noindent\textit{Materai Rp 10.000}

\end{document}
